\chapter{Total aerial volume\label{chap::total_v}}

In this section, I explore two possible paths to estimate the total aerial volume, \( \Vtot \). The first option, based on \cite{Longuetaud2013}, uses the ratio:
\begin{equation}
	r = \frac{\Vbole}{\Vtot} \label{eq::r_boletot}
\end{equation}
and is comprised between 0 and 1 by construction. Therefore, a Beta-distribution is a good fit. The second option is to fit a multivariate normal distribution, \( \MVN \), to the \textit{transformed} ratio, \( \logit(r) \), jointly with the log-transformed total volume, \( \log(\Vtot) \):
\begin{equation}
	\begin{pmatrix}
		\logit(r) \\
		\log(\Vtot)
	\end{pmatrix}
	\sim
	\MVN\left\{ \begin{pmatrix}
		f(c, \, h; \symbfup{\theta}) \\
		g(c, \, h; \symbfup{\theta}) \\
	\end{pmatrix}, \Sigmabf \right\},
	\label{eq::mvn}
\end{equation}
where \( f \) and \( g \) are functions tailored to our specific needs, \( c \) is circumference at breast height, \( h \), is tree height, and \( \theta \) is a vector of parameters to estimate.

\section{Ratio approach}

Put here the results to compare

\section{Multivariate approach}

Multivariate models allow pulling information across components, \ie they describe the variation within and covariation among tree components. I chose to model jointly \( r \) and \( \Vtot \) but the alternative modelling bole and crown could also be considered.

\begin{tcolorbox}[breakable, title = Intractability]
Note that whatever I try to model, there will always be intractability somewhere. If I model both components -- trunk and crown -- with a multivariate lognormal, the distribution of the total volume becomes untractable (sum of two lognormals has no closed form distribution). Alternatively, if I model the total volume and the ratio \( r \), then \( \Vbole \) becomes untractable. Indeed:
\begin{align*}
	\E[\Vbole] &= \E[r \Vtot] \\
	\E[\Vbole] &= \E[r] \E[\Vtot] + \Cov[r, \Vtot]
\end{align*}
and similarly for the variance. In any case, I do not have a closed-form distribution.
\end{tcolorbox}

\subsection{Simulated data}
In order to demonstrate the relevance of multivariate methods, I simulate two correlated random variables and then recover the generating parameters. Let \( X \) be an explanatory variable, and \( \mathbf{Y} = (Y_1, \, Y_2) \) be the dependant variable, such that:
\[
	\begin{pmatrix}
		Y_1 \\
		Y_2
	\end{pmatrix} \sim
	\MVN\left\{ \begin{pmatrix}
		\alpha_1 + \beta_1 X \\
		\alpha_2 + \beta_2 X \\
	\end{pmatrix}, \Sigmabf \right\},
\]
where \( \alpha \) and \( \beta \) are regression parameters to estimate, and \( \Sigmabf \) is the variance-covariance matrix defined by:
\[
	\Sigmabf = \begin{bmatrix}
		\sigma_1^2 & \rho \sigma_1 \sigma_2 \\
		\rho \sigma_1 \sigma_2 & \sigma_2^2
	\end{bmatrix}. \\
\]

I am able to recover the parameters \( \alpha \), \( \beta \) and the variance-covariance matrix \( \Sigmabf \) with both a multivariate normal or two separated linear regression as shown in Fig \ref{fig::res-params}. However, when multiplying both variables \( Y_1 \) and \( Y_2 \), as we would do to compute the bole volume from the total volume and the ratio \( r \), we see that the two separated linear models give a biased result with a smaller credibility interval (see Figs. \ref{fig::res-15} -- \ref{fig::res-85}). This is because the real expected value for the product is \( \E[Y_1 Y_2] = \E[Y_1] \E[Y_2] + \Cov[Y_1, Y_2] \) in the correlated case, while for the uncorrelated case it assumes \( \E[Y_1 Y_2] = \E[Y_1] \E[Y_2] \) (\ie a covariance of \num{0}). The R code is available on \href{https://github.com/amael-ls/emerge/tree/main/communications/cst/2025-10-30/code}{\faIcon{github} Github}. \\

In this simulated case, the principal drawback of employing a \textit{classical} rather than a \textit{multivariate} model is that it produces narrower credibility intervals and apparently consistent parameter estimates, yet leads to a biased output product \( Y_1 Y_2\).

\begin{figure}[htb]
	\centering
	\setkeys{Gin}{width=\linewidth}
	\begin{subfigure}{0.4\textwidth}
		\includegraphics{./Figures/alpha_corr.pdf}
		\caption{Multivariate model}
	\end{subfigure}
	\hfil
	\begin{subfigure}{0.4\textwidth}
		\includegraphics{./Figures/alpha_uncorr.pdf}
		\caption{Uncorrelated model}
	\end{subfigure}
	\caption{Posterior predictions of the product \( Y_1 Y_2 \) for a value sitting around the 15\textup{th} percentile}
	\label{fig::res-params}
\end{figure}


\begin{figure}[htb]
	\centering
	\setkeys{Gin}{width=\linewidth}
	\begin{subfigure}{0.4\textwidth}
		\includegraphics{./Figures/15th-percentile_corr.pdf}
		\caption{Multivariate model}
	\end{subfigure}
	\hfil
	\begin{subfigure}{0.4\textwidth}
		\includegraphics{./Figures/15th-percentile_uncorr.pdf}
		\caption{Uncorrelated model}
	\end{subfigure}
	\caption{Posterior predictions of the product \( Y_1 Y_2 \) for a value sitting around the 15\textup{th} percentile}
	\label{fig::res-15}
\end{figure}

\begin{figure}[htb]
	\centering
	\setkeys{Gin}{width=\linewidth}
	\begin{subfigure}{0.4\textwidth}
		\includegraphics{./Figures/50th-percentile_corr.pdf}
		\caption{Multivariate model}
	\end{subfigure}
	\hfil
	\begin{subfigure}{0.4\textwidth}
		\includegraphics{./Figures/50th-percentile_uncorr.pdf}
		\caption{Uncorrelated model}
	\end{subfigure}
	\caption{Posterior predictions of the product \( Y_1 Y_2 \) for a value sitting around the 50\textup{th} percentile (\ie median)}
	\label{fig::res-50}
\end{figure}

\begin{figure}[htb]
	\centering
	\setkeys{Gin}{width=\linewidth}
	\begin{subfigure}{0.4\textwidth}
		\includegraphics{./Figures/85th-percentile_corr.pdf}
		\caption{Multivariate model}
	\end{subfigure}
	\hfil
	\begin{subfigure}{0.4\textwidth}
		\includegraphics{./Figures/85th-percentile_uncorr.pdf}
		\caption{Uncorrelated model}
	\end{subfigure}
	\caption{Posterior predictions of the product \( Y_1 Y_2 \) for a value sitting around the 85\textup{th} percentile}
	\label{fig::res-85}
\end{figure}

\subsection{Real case study}

I tested a simple multivariate model on \( \logit(r) \) and \( \log(\Vtot) \) with only two intercepts and \( \Sigmabf \) to estimate. I found a positive correlation of 0.28 (0.24 -- 0.31) for \textit{Fagus sylvatica} between the two, suggesting that larger total volumes are predominantly concentrated in the bole.
