\chapter{Total aerial volume\label{chap::total_v}}

In this section, I explore two possible paths to estimate the total aerial volume, \( \Vtot \). The first option, based on \cite{Longuetaud2013}, uses the ratio:
\begin{equation}
	r = \frac{\Vbole}{\Vtot} \label{eq::r_boletot}
\end{equation}
and is comprised between 0 and 1 by construction. Therefore, a Beta-distribution is a good fit. The second option is to fit a multivariate normal distribution, \( \MVN \), to the \textit{transformed} ratio, \( \logit(r) \), jointly with the log-transformed total volume, \( \log(\Vtot) \):
\begin{equation}
	\begin{pmatrix}
		\logit(r) \\
		\log(\Vtot)
	\end{pmatrix}
	\sim
	\MVN\left\{ \begin{pmatrix}
		f(c, \, h; \symbfup{\theta}) \\
		g(c, \, h; \symbfup{\theta}) \\
	\end{pmatrix}, \symbfup{\Sigma} \right\},
	\label{eq::mvn}
\end{equation}
where \( f \) and \( g \) are functions to tailor for our specific needs, \( c \) is circumference at breast height, \( h \), is tree height, and \( \theta \) is a vector of parameters to estimate.

\section{Ratio approach}

\section{Multivariate approach}

\subsection{Simulated data}
In order to demonstrate the relevance of multivariate methods, I simulate two correlated random variables and then recover the generating parameters.