\chapter{Context\label{chap::context}}

The French National Forest Inventory (\NFI) publishes statistics annually, such as basal area or wood resources, on both private and public forest. These statistics are derived from a sampling scheme in which about \num{6000} new plots are surveyed each year, along with \num{6000} plots that were first visited five years earlier. The published statistics are calculated using a moving five-year average; for instance, the published results for 2024 are based on campaigns spanning from 2020 to 2024. \\

Initially designed to assess forest area and produce estimates of standing timber stock, the French \NFI{} is gradually evolving into more comprehensive tools for forest monitoring, incorporating a variety of indicators. Among this new information, the estimation of forest biomass and carbon is crucial, as it makes it possible to monitor biomass production and its use by different sectors, to estimate the contribution of forests to mitigating the effects of climate change as part of climate commitments monitoring (Citepa), and to develop forest strategies adapted to contemporary environmental and societal challenges \parencite{Commission2018}. \\

% Check \sidecite, and maybe tweak it to put only the author and title in the margin

The European Union developed a \href{https://eur-lex.europa.eu/legal-content/EN/TXT/?uri=CELEX:52021DC0572}{New EU Forest Strategy for 2030} as part of the plan to adapt to and fight against climate change and make Europe a climate neutral continent by 2050. This strategy relies on improved monitoring of European forests to better understand their condition and respond accordingly. Specifically, it calls for assessing carbon sequestration in forests to evaluate whether or not Eu\-ro\-pe reached carbon neutrality. One bottleneck is the harmonisation of the forest monitoring methods between European member states, if not within them. The \href{https://pathfinder-heu.eu/#top}{PathFinder project} supports member states in implementing a European Forest Monitoring System in order to standardise or harmonise forest data collection and reporting across the EU. This prompted the French \NFI{} to update its methods for assessing forest carbon storage. \\

Three steps are necessary to estimate carbon storage from field data (see Fig. \ref{fig::scheme}): (\textit{i}) the total above-ground volume is estimated from diameter at breast height (dbh) and height \parencite{Vallet2006}, (\textit{ii}) biomass is then derived by using a coefficient for wood density, and (\textit{iii}) a factor is applied to convert the biomass into carbon content. \\

\begin{figure}[h]
    \centering
	\begin{tikzpicture}
	\node (vol) at (0, 0) {Volume};
	\node[below = of vol.mid east, anchor = mid east] (dens) {Density};

	\path (vol.mid east) -- (dens.mid east) node[midway, right = 0.8cm, anchor = mid west] (bio) {Biomass};
	\draw[->, arrow, out = 0, in = 180] (vol.mid east) to (bio.mid west);
	\draw[->, arrow, out = 0, in = 180] (dens.mid east) to (bio.mid west);

	\node[right = 0.8cm of bio.mid east, anchor = mid west] (C) {Carbon};
	\draw[->, arrow] (bio.mid east) to (C.mid west);

	\node[database, above left = 0.3cm of vol, label = above:Emerge, database radius = 0.3cm,
		database segment height = 0.15cm] (emerge) {};

	\node[database, below left = 0.3cm of dens, label = below:XyloDensMap, database radius = 0.3cm,
		database segment height = 0.15cm] (xdm) {};

	\draw[->, arrow, dashed, out = -90, in = 180] (emerge.south) to (vol.mid west);
	\draw[->, arrow, dashed, out = 90, in = 180] (xdm.north) to (dens.mid west);
\end{tikzpicture}
	\caption{Computation chain for biomass and carbon content.\label{fig::scheme}}
\end{figure}

In this technical note, we explain the whole computation chain we foresee to estimate carbon content. We focus mostly on the base of the chain, \ie the two volumes computed at the French \NFI{} for each tree (bole and total volumes), and the wood density. In the section \nameref{chap::def}, we define the different tree components and explain the origins of the data. Then, in the sections \nameref{chap::bole_v} and \nameref{chap::total_v}, we expose the models to compute the individual bole and total volumes, respectively. Lastly, we present the undergoing work on wood density and potential path to convert total above-ground volume into biomass and carbon content in the sections \nameref{chap::xdm} and \nameref{chap::conclusion}.
