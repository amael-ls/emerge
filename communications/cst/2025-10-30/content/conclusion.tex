\chapter{Conclusion\label{chap::conclusion}}

\section{Above-ground volumes (bole and total)}
Two approaches can be explored here. The first one is to separate the estimation of 


\section{Root volume and wood carbon content}

While actions regarding aboveground volumes (stem wood and total above-ground biomass) and wood density are already underway, aspects related to root volume and carbon content in wood remain less explored at this stage. Unlike the evaluation of or wood density, the IGN does not currently have datasets that allow for the analysis of root volumes or wood carbon content. For these aspects, a different approach will be implemented, involving a literature review to assess the current state of knowledge. \\

The objective will be to determine whether the current assumptions used in the IGN method---namely, a root expansion coefficient by botanical class (1.28 for broadleaves and 1.30 for conifers) and a single carbon content coefficient (0.475)---are still relevant in light of current knowledge, or whether they could be improved to better reflect natural variability. The literature review will be complemented by discussions with expert researchers in these fields, such as those at the INRAE center in Champenoux. \\

An internship was already conducted during the summer of 2025 on this topic. It helped identify a number of bibliographic references for each aspect. Regarding wood carbon content, many articles have recently been published, and a global database was even compiled through a literature review \parencite{Doraisami2022}. This database includes carbon content values for several species found in French forests. As for root volume, an interview was conducted with Frédéric Danjon, a root specialist at INRAE, which provided access to numerous relevant references on root volume. A synthesis of the various identified references for each aspect now needs to be carried out. \\

Should the assumptions regarding root volume and carbon content be revised, an analysis and documentation of the impact of these revisions on carbon estimates will be undertaken via a sensitivity analysis \parencite{Iooss2017}.