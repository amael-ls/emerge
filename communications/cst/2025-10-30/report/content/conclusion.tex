\chapter{Conclusion\label{chap::conclusion}}

\section{Above-ground volumes (bole and total)}
We presented two alternative approaches for estimating tree volumes. The first approach involves estimating \( \Vbole \) and \( \Vtot \) separately, using a ratio for the total above-ground volume to ensure consistency between both volumes. Indeed, the main problem so far has been that the formula currently used by the French \NFI{} to predict \( \Vbole \) \parencite{Morneau2016} sometimes yields values greater than the total volume predicted by \cite{Vallet2006}. Using the first approach with the new allomettries for \( \Vbole \) has the advantage of retaining the existing \NFI{} bole volume while providing a coherent total volume such that \( \Vtot > \Vbole \). The newly developed allometric equation for \( \Vbole \) yields results consistent with previous time series. \\

The second approach has the advantage of jointly fitting both volumes, thereby accounting for correlations. As shown in the simulated data, disregarding correlations can strongly bias estimates of the product of two correlated random variables. As for the first alternative, the multivariate approach ensures consistent volumes by construction, but is harder to parametrise and entails a loss of analytical tractability due to the product of two (correlated) random variables. Refitting the bole volume might also imply a break in the time series generated by the French \NFI{} since 1958, which may require retaining both the old and new allometries for compatibility reasons with external studies. \\

A third option, not yet explored, would be to use \textit{copulas} \parencite{Nelsen2006}. Copulas also belong to the family of multivariate models, and could be described as functions that join or couple multivariate distribution functions to their one-dimensional marginal distribution functions. In other words, it would allow us to model \( r \) directly with a Beta-distribution and \( \Vtot \) with, for example, a Gamma-distribution rather than relying on a multivariate normal model in which all components of the random vector must follow a normal distribution. However, copulas are more complicated and would require a dedicated review of two-dimensional copulas (\textcite{Nelsen2006} provides a well written introduction covering all aspects relevant to our application).

\section{Root volume and wood carbon content}

While actions regarding aboveground volumes (stem wood and total above-ground biomass) and wood density are already underway, aspects related to root volume and carbon content in wood remain less explored at this stage. Unlike the evaluation of or wood density, the IGN does not currently have datasets that allow for the analysis of root volumes or wood carbon content. For these aspects, a different approach will be implemented, involving a literature review to assess the current state of knowledge. \\

The objective will be to determine whether the current assumptions used in the IGN method---namely, a root expansion coefficient by botanical class (1.28 for broadleaves and 1.30 for conifers) and a single carbon content coefficient (0.475)---are still relevant in light of current knowledge, or whether they could be improved to better reflect natural variability. The literature review will be complemented by discussions with expert researchers in these fields, such as those at the INRAE center in Champenoux. \\

An internship was already conducted during the summer of 2025 on this topic. It helped identify a number of bibliographic references for each aspect. Regarding wood carbon content, many articles have recently been published, and a global database was even compiled through a literature review \parencite{Doraisami2022}. This database includes carbon content values for several species found in French forests. As for root volume, an interview was conducted with Frédéric Danjon, a root specialist at INRAE, which provided access to numerous relevant references on root volume. A synthesis of the various identified references for each aspect now needs to be carried out.

\section{Final words}

Regardless of the chosen approach, the next step will be to quantify and document the impact of these revisions on biomass and carbon content estimates through a sensitivity analysis \parencite{Iooss2017} with respect to volumes and densities.

% Should the assumptions regarding root volume and carbon content be revised, an analysis and documentation of the impact of these revisions on carbon estimates will be undertaken via a sensitivity analysis \parencite{Iooss2017}.